	\chapter{Introduction}
	\section{Axioms}
	The format of the axioms is that on the first line you will find a definition of the axiom written in predicate logic. 
	On the following line or lines you find the definition expressed in a language more closely to 
	what is used in daily speach. This is less precise than the formal definition in the first line, but of cause easier to read and understand.
	\begin{enumerate}
		\item $\mathbb{C} \supset \Re$ \\ The set of complex numbrs are a proper superset of all real numbers.
		\item $\forall s \in \mathbb{C} ( s = a + ib \wedge i = \sqrt{-1} \wedge a \in \Re \wedge b \in \Re)$ \\ It is true that for all numbers s in the set $\mathbb{C}$ that $ i = \sqrt{-1}$ and that a is a real number and that b is a real number.
		\item $\forall s \in \mathbb{C}$ (Re(s) = a $\wedge$ Im(s) = b) \\ The function Re applied to any complex number s will return the real part a, and the function Im will return the imaginary part b.
		\item $ ib \in s \perp a \in s $ \\ All values bi exists in a dimension perpendicular to $\Re$. All values in $\mathbb{C}$ exists in a plane, where one axis is the real coordinate and the imaginary component of s is regarded as ocupying another coordinate axis in a cartesian coordinate system.
		\item $\exists s \in \mathbb{C} (s = 0 \wedge s = 0 + 0i) $ \\ There exists an elemment s in $\mathbb{C}$ such that s is null and the meaning of this is that a = 0 and b = 0. This is called the null element.
		\item $\exists s \in \mathbb{C} (s = 1 \wedge s = 1+0i) $ \\ There exists an element s in $\mathbb{C}$ such that s is 1 and the meaning of this is that a = 1 and b = 0. This is 
	\end{enumerate}
	
	\section{Different ways to express a complex number}